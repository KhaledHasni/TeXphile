% ~~~~~~~~~~~~~~~~~~~~~~~~~~~~~~~~~~~~~~ TeXphile ~~~~~~~~~~~~~~~~~~~~~~~~~~~~~~~~~~~~~~~
% TeXphile Resume template
% Designed & Created by: Khaled Hasni
% Shared & Distributed under the terms of the BUSL license
% customize.tex: Customizes the resume by adding/removing certain features
% ~~~~~~~~~~~~~~~~~~~~~~~~~~~~~~~~~~~~~~~~~~~~~~~~~~~~~~~~~~~~~~~~~~~~~~~~~~~~~~~~~~~~~~~

% ************************************ Resume Header ************************************
% 1) Personal information:
% Use these string literals to set your personal information (Full name, Job title, Phone number, Email address, Linkedin profile, Github profile).
% This information will be displayed in the resume header.
% Notes: * Full name and job title are mandatory.
%        * Phone number, Email address, Linkedin profile and Github profile are all optional. Use the toggle switches below to remove any of them if you don't wish to have them in your resume header.
\def\MyFullName{Khaled Hasni}
\def\MyJobTitle{Software Engineer | Firmware Consultant}
\def\MyPhoneNumber{+x xx xxx xxx}
\def\MyEmailAddress{Not.Khaled.Hasni@gmail.com}
\def\MyLinkedinProfile{linkedin.com/in/khaledhasni}
\def\MyGithubProfile{github.com/KhaledHasni}

% 2) Toggle switches:
% Use these to toggle some of the resume header features on and off.
% y --> On
% n --> Off
% Examples:
% \newcommand{\SmallCaps}{n} to have non-initial letters lowercased in your full name 
% \newcommand{\PhoneNbr}{n} to remove the phone number entry from the resume header 
% \newcommand{\EmailAddr}{n} to remove the email address entry from the resume header
% \newcommand{\LinkedinProf}{n} to remove the Linkedin profile entry from the resume header
% \newcommand{\GithubProf}{n} to remove the Github profile entry from the resume header
% \newcommand\UseIcons{n} to remove all icons (phone, email, linkedin, github)
\newcommand{\SmallCaps}{y}
\newcommand{\PhoneNbr}{y}
\newcommand{\EmailAddr}{y}
\newcommand{\LinkedinProf}{y}
\newcommand{\GithubProf}{y}
\newcommand{\UseIcons}{y}

% *************************** Professional Experience Section ***************************

% 1) Toggle switches:
% Use this to keep/remove the Professional Experience section.
% y --> Keep Professional Experience section
% n --> Remove Professional Experience section
\newcommand{\ProfessionalExperience}{y}

% Use these toggle switches to add/take away a professional experience.
% y --> Keep professional experience
% n --> Remove professional experience
% Example: If you wish to mention 3 professional experiences, set the 1st, 2nd and 3rd toggle switches and clear the rest:
% \newcommand{\EnableFirstProfessionalExperience}{y}
% \newcommand{\EnableSecondProfessionalExperience}{y}
% \newcommand{\EnableThirdProfessionalExperience}{y}
% \newcommand{\EnableFourthProfessionalExperience}{n}
% \newcommand{\EnableFifthProfessionalExperience}{n}
% \newcommand{\EnableSixthProfessionalExperience}{n}
% \newcommand{\EnableSeventhProfessionalExperience}{n}
% \newcommand{\EnableEighthProfessionalExperience}{n}
% \newcommand{\EnableNinthProfessionalExperience}{n}
% \newcommand{\EnableTenthProfessionalExperience}{n}
%
% Note: A maximum of 10 experiences is supported.
\newcommand{\EnableFirstProfessionalExperience}{y}
\newcommand{\EnableSecondProfessionalExperience}{y}
\newcommand{\EnableThirdProfessionalExperience}{y}
\newcommand{\EnableFourthProfessionalExperience}{n}
\newcommand{\EnableFifthProfessionalExperience}{n}
\newcommand{\EnableSixthProfessionalExperience}{n}
\newcommand{\EnableSeventhProfessionalExperience}{n}
\newcommand{\EnableEighthProfessionalExperience}{n}
\newcommand{\EnableNinthProfessionalExperience}{n}
\newcommand{\EnableTenthProfessionalExperience}{n}

% 2) Professional Experience listings:
% Use this to enumerate the different professional experiences you have. A maximum of 10 experiences is supported.
% Every experience should contain at most 4 items: {Institute}/{Timespan}/{Role}/{Location}.
% Removing an item is possible. To do that, make sure to leave an empty set of curly brackets {}. Example: {Institute}/{}/{Role}/{Location} --> no timespan will show up in this entry.
% Don't forget to add/remove the comma ',' seperator if you want to add/remove entries.
% !!! IMPORTANT NOTE: Only the amount of professional experiences set by the toggle switches above will be considered. i.e if only 3 toggle switches are set and 4 entries are defined here, only the first 3 will be considered !!!
\newcommand{\ProfessionalExperienceEntries}
{
    % 1st professional experience
    {SoftAtHome}/{February 2024 -- Present}/{Embedded Linux Engineer}/{Wijgmaal, Belgium (Remote)},
    % 2nd professional experience
    {Valeo}/{March 2023 -- February 2024}/{Lead Firmware Consultant}/{Paris, France (Remote)},
    % 3rd professional experience
    {Sofia Technologies}/{November 2021 -- February 2024}/{Embedded Software Engineer}/{Tunis, Tunisia}
}

% 3) Achievements and highlights of all professional experiences:
% Use these to enumerate your achievements and highlights at your different professional experiences.
% There's no limit to the number of achievements you can list for every professional experience.
%
% Notes: * If you don't wish to list any achievements, make sure to leave the command entirely empty. Example: \newcommand{\FirstProfessionalExperienceAchievements}{} --> No achievements will show up for your 1st professional experience.
%        * Don't forget to add/remove the comma ',' seperator if you want to add/remove entries.
%        * Only the professional experiences set by the toggle switches above will be considered. i.e if only toggle switches 2, 7 and 9 are set for example, achievements in the 1st, 3rd, 4th, 5th, 6th, 8th and 10th professional experiences will not be considered.

% List 1st professional experience achievements here if applicable
\newcommand{\FirstProfessionalExperienceAchievements}
{
    {Developed a service to automatically perform a set of unit tests daily on a product in development in order to decrease time needed for team members to identify and fix bugs/issues},
    {Incorporated scripts using Python and PowerShell to aggregate XML test results into an organized format and to load the latest build code onto the hardware, so that daily testing can be performed},
    {Utilized Jenkins to provide a continuous integration service in order to automate the entire process of loading the latest build code and test files, running the tests, and generating a report of the results once per day}
}

% List 2nd professional experience achievements here if applicable
\newcommand{\SecondProfessionalExperienceAchievements}
{
    {Assisted in development of the front end of a mobile application for iOS/Android using Dart and the Flutter framework},
    {Worked with Google Firebase to manage user inputted data across multiple platforms including web and mobile apps}
}

% List 3rd professional experience achievements here if applicable
\newcommand{\ThirdProfessionalExperienceAchievements}
{
    {Collaborated with team members using version control systems such as Git to organize modifications and assign tasks},
    {Utilized Android Studio as a development environment in order to visualize the application in both iOS and Android},
    {Explored ways to visualize and send a daily report of test results to team members  using HTML, Javascript, and CSS}
}

% List 4th professional experience achievements here if applicable
\newcommand{\FourthProfessionalExperienceAchievements}
{}

% List 5th professional experience achievements here if applicable
\newcommand{\FifthProfessionalExperienceAchievements}
{}

% List 6th professional experience achievements here if applicable
\newcommand{\SixthProfessionalExperienceAchievements}
{}

% List 7th professional experience achievements here if applicable
\newcommand{\SeventhProfessionalExperienceAchievements}
{}

% List 8th professional experience achievements here if applicable
\newcommand{\EighthProfessionalExperienceAchievements}
{}

% List 9th professional experience achievements here if applicable
\newcommand{\NinthProfessionalExperienceAchievements}
{}

% List 10th professional experience achievements here if applicable
\newcommand{\TenthProfessionalExperienceAchievements}
{}

% ********************************** Projects Section ***********************************

% 1) Toggle switches:
% Use this to keep/remove the Projects section.
% y --> Keep Projects section
% n --> Remove Projects section
\newcommand{\Projects}{y}

% Use these toggle switches to add/take away a project.
% y --> Keep project
% n --> Remove project
% Example: If you wish to mention 3 projects, set the 1st, 2nd and 3rd toggle switches and clear the rest:
% \newcommand{\EnableFirstProject}{y}
% \newcommand{\EnableSecondProject}{y}
% \newcommand{\EnableThirdProject}{y}
% \newcommand{\EnableFourthProject}{n}
% \newcommand{\EnableFifthProject}{n}
% \newcommand{\EnableSixthProject}{n}
% \newcommand{\EnableSeventhProject}{n}
% \newcommand{\EnableEighthProject}{n}
% \newcommand{\EnableNinthProject}{n}
% \newcommand{\EnableTenthProject}{n}
%
% Note: A maximum of 10 projects is supported.
\newcommand{\EnableFirstProject}{y}
\newcommand{\EnableSecondProject}{y}
\newcommand{\EnableThirdProject}{y}
\newcommand{\EnableFourthProject}{n}
\newcommand{\EnableFifthProject}{n}
\newcommand{\EnableSixthProject}{n}
\newcommand{\EnableSeventhProject}{n}
\newcommand{\EnableEighthProject}{n}
\newcommand{\EnableNinthProject}{n}
\newcommand{\EnableTenthProject}{n}

% 2) Projects listings:
% Use this to enumerate the different projects you have. A maximum of 10 projects is supported.
% Every project should contain at most 3 items: {ProjectName}/{Keywords}/{Timespan}.
% Removing an item is possible. To do that, make sure to leave an empty set of curly brackets {}. Example: {ProjectName}/{}/{Timespan} --> no keywords will show up in this entry.
% Don't forget to add/remove the comma ',' seperator if you want to add/remove entries.
% !!! IMPORTANT NOTE: Only the amount of projects set by the toggle switches above will be considered. i.e if only 3 toggle switches are set and 4 entries are defined here, only the first 3 will be considered !!!
\newcommand{\ProjectsEntries}
{
    % 1st project 
    {Gym Reservation Bot}/{Python, Selenium, Google Cloud Console}/{January 2021},
    % 2nd project
    {Ticket Price Calculator App}/{Java, Android Studio}/{November 2020},
    % 3rd project
    {Transaction Management GUI}/{Java, Eclipse, JavaFX}/{October 2020}
}

% 3) Achievements and highlights of all projects:
% Use these to enumerate your achievements and highlights in your different projects.
% There's no limit to the number of achievements you can list for every project.
%
% Notes: * If you don't wish to list any achievements, make sure to leave the command entirely empty. Example: \newcommand{\FirstProjectAchievements}{} --> No achievements will show up for your 1st project.
%        * Don't forget to add/remove the comma ',' seperator if you want to add/remove entries.
%        * Only the projects set by the toggle switches above will be considered. i.e if only toggle switches 2, 7 and 9 are set for example, achievements in the 1st, 3rd, 4th, 5th, 6th, 8th and 10th projects will not be considered.

% List 1st project achievements here if applicable
\newcommand{\FirstProjectAchievements}
{
    {Developed an automatic bot using Python and Google Cloud Console to register myself for a timeslot at my school gym},
    {Implemented Selenium to create an instance of Chrome in order to interact with the correct elements of the web page},
    {Created a Linux virtual machine to run on Google Cloud so that the program is able to run everyday from the cloud},
    {Used Cron to schedule the program to execute automatically at 11 AM every morning so a reservation is made for me}
}

% List 2nd project achievements here if applicable
\newcommand{\SecondProjectAchievements}
{
    {Created an Android application using Java and Android Studio to calculate ticket prices for trips to museums in NYC},
    {Processed user inputted information in the back-end of the app to return a subtotal price based on the tickets selected},
    {Utilized the layout editor to create a UI for the application in order to allow different scenes to interact with each other}
}

% List 3rd project achievements here if applicable
\newcommand{\ThirdProjectAchievements}
{
    {Designed a sample banking transaction system using Java to simulate the common functions of using a bank account},
    {Used JavaFX to create a GUI that supports actions such as creating an account, deposit, withdraw, list all acounts, etc},
    {Implemented object-oriented programming practices such as inheritance to create different account types and databases}
}

% List 4th project achievements here if applicable
\newcommand{\FourthProjectAchievements}
{}

% List 5th project achievements here if applicable
\newcommand{\FifthProjectAchievements}
{}

% List 6th project achievements here if applicable
\newcommand{\SixthProjectAchievements}
{}

% List 7th project achievements here if applicable
\newcommand{\SeventhProjectAchievements}
{}

% List 8th project achievements here if applicable
\newcommand{\EighthProjectAchievements}
{}

% List 9th project achievements here if applicable
\newcommand{\NinthProjectAchievements}
{}

% List 10th project achievements here if applicable
\newcommand{\TenthProjectAchievements}
{}

% ************************************ Skills Section *************************************

% 1) Toggle switch:
% Use this to keep/remove the Skills section.
% y --> Keep Skills section
% n --> Remove Skills section
\newcommand{\Skills}{y}

% 2) Skills listings:
% Use this to enumerate the different skills you have.
% A skill entry is made up of 2 items: {Skill category}/{Individual skills}
% There's no limit on the number of entries you can list here.
% Don't forget to add/remove the comma ',' seperator if you want to add/remove entries.
\newcommand{\SkillsEntries}
{
    {Lanuguages}/{C, Python, Shell},
    {Developer Tools}/{Docker, Kubernetes, Klocwork, Valgrind},
    {Unit Testing tools \& Frameworks}/{VectorCAST, Cmocka, Unity},
    {Revision Control}/{Git, Mercurial},
    {Build \& Deployment}/{Yocto, Make, PlatformIO}
}

% ********************************** Education Section **********************************

% 1) Toggle switches:
% Use this to keep/remove the Education section.
% y --> Keep Education section
% n --> Remove Education section
\newcommand{\Education}{y}

% Use these toggle switches to add/take away an education entry.
% y --> Keep education entry
% n --> Remove education entry
% Example: If you wish to mention 3 education entries, set the 1st, 2nd and 3rd toggle switches and clear the rest:
% \newcommand{\EnableFirstEducation}{y}
% \newcommand{\EnableSecondEducation}{y}
% \newcommand{\EnableThirdEducation}{y}
% \newcommand{\EnableFourthEducation}{n}
% \newcommand{\EnableFifthEducation}{n}
% \newcommand{\EnableSixthEducation}{n}
% \newcommand{\EnableSeventhEducation}{n}
% \newcommand{\EnableEighthEducation}{n}
% \newcommand{\EnableNinthEducation}{n}
% \newcommand{\EnableTenthEducation}{n}
%
% Note: A maximum of 10 entries is supported.
\newcommand{\EnableFirstEducation}{y}
\newcommand{\EnableSecondEducation}{y}
\newcommand{\EnableThirdEducation}{y}
\newcommand{\EnableFourthEducation}{n}
\newcommand{\EnableFifthEducation}{n}
\newcommand{\EnableSixthEducation}{n}
\newcommand{\EnableSeventhEducation}{n}
\newcommand{\EnableEighthEducation}{n}
\newcommand{\EnableNinthEducation}{n}
\newcommand{\EnableTenthEducation}{n}

% 2) Education listings:
% Use this to enumerate the different education experiences you have. A maximum of 10 entries is supported.
% Every education experience should contain at most 4 items: {Institute}/{Timespan}/{Coursework}/{Location}.
% Removing an item is possible. To do that, make sure to leave an empty set of curly brackets {}. Example: {Institute}/{Timespan}/{Coursework}/{} --> no location will show up in this entry.
% Don't forget to add/remove the comma ',' seperator if you want to add/remove entries.
% !!! IMPORTANT NOTE: Only the amount of education experiences set by the toggle switches above will be considered. i.e if only 3 toggle switches are set and 4 entries are defined here, only the first 3 will be considered !!!
\newcommand{\EducationEntries}
{
    % Most recent education experience 
    {ENIT}/{March 2023 -- February 2024}/{Electrical Engineering degree}/{Tunis, Tunisia},
    % 2nd most recent education experience
    {IPEST}/{February 2024 -- Present}/{Maths/Physics}/{Tunis, Tunisia},
    % 3rd most recent education experience
    {Lycee Pilote Ariana}/{November 2021 -- February 2024}/{Highschool degree}/{Tunis, Tunisia}
}

% 3) Achievements and highlights of all education experiences:
% Use these to enumerate your achievements and highlights in your education experiences.
% There's no limit to the number of achievements you can list for every education experience.
%
% Notes: * If you don't wish to list any achievements, make sure to leave the command entirely empty. Example: \newcommand{\FirstEducationAchievements}{} --> No achievements will show up for your 1st education experience.
%        * Don't forget to add/remove the comma ',' seperator if you want to add/remove entries.
%        * Only the education experiences set by the toggle switches above will be considered. i.e if only toggle switches 2, 7 and 9 are set for example, achievements in the 1st, 3rd, 4th, 5th, 6th, 8th and 10th education experiences will not be considered.

% List 1st education achievements here if applicable
\newcommand{\FirstEducationAchievements}
{}

% List 2nd education achievements here if applicable
\newcommand{\SecondEducationAchievements}
{}

% List 3rd education achievements here if applicable
\newcommand{\ThirdEducationAchievements}
{}

% List 4th education achievements here if applicable
\newcommand{\FourthEducationAchievements}
{}

% List 5th education achievements here if applicable
\newcommand{\FifthEducationAchievements}
{}

% List 6th education achievements here if applicable
\newcommand{\SixthEducationAchievements}
{}

% List 7th education achievements here if applicable
\newcommand{\SeventhEducationAchievements}
{}

% List 8th education achievements here if applicable
\newcommand{\EighthEducationAchievements}
{}

% List 9th education achievements here if applicable
\newcommand{\NinthEducationAchievements}
{}

% List 10th education achievements here if applicable
\newcommand{\TenthEducationAchievements}
{}