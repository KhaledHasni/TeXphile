% ~~~~~~~~~~~~~~~~~~~~~~~~~~~~~~~~~~~~~~ TeXfolio ~~~~~~~~~~~~~~~~~~~~~~~~~~~~~~~~~~~~~~~
% TeXfolio Resume template
% Designed & Created by: Khaled Hasni
% Shared & Distributed under the terms of the MIT license
% customize.tex: Customizes the resume by adding/removing certain features
% ~~~~~~~~~~~~~~~~~~~~~~~~~~~~~~~~~~~~~~~~~~~~~~~~~~~~~~~~~~~~~~~~~~~~~~~~~~~~~~~~~~~~~~~

% ************************************ Resume Header ************************************
% 1) Personal information:
% Use these string literals to set your personal information (Full name, Job title, Phone number, Email address, Linkedin profile, Github profile).
% This information will be displayed in the resume header.
\def\MyFullName{Khaled Hasni}
\def\MyJobTitle{Software Engineer | Firmware Consultant}
\def\MyPhoneNumber{+x xx xxx xxx}
\def\MyEmailAddress{Not.Khaled.Hasni@gmail.com}
\def\MyLinkedinProfile{linkedin.com/in/khaledhasni}
\def\MyGithubProfile{github.com/KhaledHasni}

% 2) Toggle switches:
% Use these to toggle some of the resume header features on and off
% y --> On
% n --> Off
% Example:
% \newcommand\UseIcons{n} to remove all icons (phone, email, linkedin, github)
\newcommand{\SmallCaps}{y}
\newcommand{\PhoneNbr}{y}
\newcommand{\EmailAddr}{y}
\newcommand{\LinkedinProf}{y}
\newcommand{\GithubProf}{y}
\newcommand{\UseIcons}{y}

% *************************** Professional Experience Section ***************************

% 1) Toggle switches:
% y --> Keep Professional Experience section
% n --> Remove Professional Experience section
\newcommand{\ProfessionalExperience}{y}

% 2) Professional Experience listings:
% Use this to enumerate the different professional experiences you had. There's no limit on the number of experiences you can list.
% Every experience should contain at most 4 entries (Institute, Timespan, Role, Location).
% Don't forget the comma ',' seperator if you add/remove entries.
% Removing an entry is possible. To do that, make sure to leave an empty set of curly brackets {}.
\newcommand{\ProfessionalExperienceEntries}
{
    % Most recent professional experience 
    {SoftAtHome}/{February 2024 -- Present}/{Embedded Linux Engineer}/{Wijgmaal, Belgium (Remote)},
    % 2nd most recent professional experience
    {Valeo}/{March 2023 -- February 2024}/{Lead Firmware Consultant}/{Paris, France (Remote)},
    % 3rd most recent professional experience
    {Sofia Technologies}/{November 2021 -- February 2024}/{Embedded Software Engineer}/{Tunis, Tunisia}
}

\newcommand{\ProfessionalExperienceAchievements}
{{khaled,maha,neji},{bachir,ahlem,hedia},{ali,hakima,hsan}}

% ********************************** Projects Section ***********************************

% 1) Toggle switches:
% y --> Keep Projects section
% n --> Remove Projects section
\newcommand{\Projects}{y}

% 2) Projects listings:
% Use this to enumerate the different projects you have. There's no limit on the number of projects you can list.
% Every project should contain at most 3 entries (Name, Keywords, Timespan).
% Don't forget the comma ',' seperator if you add/remove entries.
% Removing an entry is possible. To do that, make sure to leave an empty set of curly brackets {}.
\newcommand{\ProjectsEntries}
{
    % 1st project 
    {Gym Reservation Bot}/{Python, Selenium, Google Cloud Console}/{January 2021},
    % 2nd project
    {Ticket Price Calculator App}/{Java, Android Studio}/{November 2020},
    % 3rd project
    {Transaction Management GUI}/{Java, Eclipse, JavaFX}/{October 2020}
}


% ************************************ Skills Section *************************************

% 1) Toggle switches:
% y --> Keep Skills section
% n --> Remove Skills section
\newcommand{\Skills}{y}

% Use these toggle switches to keep/remove the different skill sets.
% y --> Keep skill set
% n --> Remove skill set
\newcommand{\Languages}{y}
\newcommand{\DeveloperTools}{y}
\newcommand{\UnitTestingAndFrameworks}{y}
\newcommand{\RevisionControl}{y}
\newcommand{\BuildAndDeployment}{y}

% ********************************** Education Section **********************************

% 1) Toggle switches:
% y --> Keep Education section
% n --> Remove Education section
\newcommand{\Education}{y}

% 2) Education listings:
% Use this to enumerate the education experiences you had. There's no limit on the number of experiences you can list.
% Every experience should contain at most 4 entries (Institute, Timespan, Coursework, Location).
% Don't forget the comma ',' seperator if you add/remove entries.
% Removing an entry is possible. To do that, make sure to leave an empty set of curly brackets {}.
\newcommand{\EducationEntries}
{
    % Most recent education experience 
    {ENIT}/{March 2023 -- February 2024}/{Electrical Engineering degree}/{Tunis, Tunisia},
    % 2nd most recent education experience
    {IPEST}/{February 2024 -- Present}/{Maths/Physics}/{Tunis, Tunisia},
    % 3rd most recent education experience
    {Lycee Pilote Ariana}/{November 2021 -- February 2024}/{Highschool degree}/{Tunis, Tunisia}
}